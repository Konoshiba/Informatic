\documentclass[8pt, letterpaper]{article}
\usepackage[utf8x]{inputenc}
\usepackage[russian]{babel}
\usepackage{graphicx}
\graphicspath{{pictures/}}
\usepackage{multicol}
\usepackage[margin=0.5in]{geometry}
\usepackage{tikz}
\usetikzlibrary {angles,quotes}
\usetikzlibrary {calc}
\usepackage{mathtools}
\usepackage{fancyhdr} 
\input{insbox}

\usepackage{fancyhdr} 

\fancypagestyle{firststyle}
{
\fancyfootoffset[R]{-12cm} 
\fancyhead[C]{КНИГА I ПРЕДЛ. XXV. ТЕОРЕМА \quad \quad \quad 49}
\renewcommand{\footrulewidth}{0.0 mm} 
\renewcommand{\headrulewidth}{0.0 mm}
\setlength{\headheight}{80pt}
\fancyfoot[R]{\thepage}
}
\begin{document}
\thispagestyle{firststyle}
\linespread{1.5}
\flushleft
\begin{multicols}{2}
\InsertBoxL{0}{\includegraphics[scale=0.6]{Screenshot_4.png}}
сли у двух треугольников две стороны
\tiny
\begin{tikzpicture}[line width=2pt]

\coordinate [label=above:$A$] (A) at (0,0);
\coordinate [label=above:$B$] (B) at (1, 0);

\draw [-, cyan] (A) -- (B);

\end{tikzpicture}
\footnotesize
и
\tiny
\begin{tikzpicture}[line width=2pt]

\coordinate [label=above:$C$] (A) at (0,0);
\coordinate [label=above:$A$] (B) at (1, 0);

\draw [-, yellow] (A) -- (B);

\end{tikzpicture}
\footnotesize
соответственно равны двум сторонам
\tiny
\begin{tikzpicture}[line width=1.5pt]

\coordinate [label=above:$D$] (A) at (0,0);
\coordinate [label=above:$E$] (B) at (1, 0);

\draw [-, cyan] (A) -- (B);

\end{tikzpicture}
\footnotesize
и
\tiny
\begin{tikzpicture}[line width=1.5pt]

\coordinate [label=above:$F$] (A) at (0,0);
\coordinate [label=above:$D$] (B) at (1, 0);

\draw [-, yellow] (A) -- (B);

\end{tikzpicture}
\footnotesize
другого, но основания неравны, то угол над большим основанием
\tiny
\begin{tikzpicture}[line width=2pt]

\coordinate [label=above:$B$] (A) at (0,0);
\coordinate [label=above:$C$] (B) at (1, 0);

\draw [-, black] (A) -- (B);

\end{tikzpicture}
\footnotesize
одного треугольника меньше угла под меньшим
\tiny
\begin{tikzpicture}[line width=1.5pt]

\coordinate [label=above:$E$] (A) at (0,0);
\coordinate [label=above:$F$] (B) at (1, 0);

\draw [-, yellow] (A) -- (B);

\end{tikzpicture}
\footnotesize
другого.\\
\setlength\parindent{100pt}
\linespread{2}
\tiny
\begin{tikzpicture}[scale=3]

  \filldraw[fill=yellow,draw=yellow] (0.08,0) -- (1.5mm,-2.6mm)
      arc [start angle=-79, end angle=-105, radius=4mm]-- cycle;
\coordinate [label=above:$A$] (B) at (0.08,0);
\coordinate [label=below:$B$] (B) at (1.5mm,-2.6mm);
\coordinate [label=below:$C$] (B) at (-0.5mm,-2.6mm);
\end{tikzpicture}
\footnotesize
= , >или <
\tiny
\begin{tikzpicture}[scale=3]

  \filldraw[fill=black,draw=black] (0.08,0) -- (1.1mm,-2.6mm)
      arc [start angle=-79, end angle=-105, radius=4mm]-- cycle;
\coordinate [label=above:$D$] (B) at (0.08,0);
\coordinate [label=below:$E$] (B) at (1.5mm,-2.6mm);
\coordinate [label=below:$F$] (B) at (-1mm,-2.6mm);
\end{tikzpicture}
\footnotesize\\
\tiny
\begin{tikzpicture}[scale=3]

  \filldraw[fill=yellow,draw=yellow] (0.08,0) -- (1.5mm,-2.6mm)
      arc [start angle=-79, end angle=-105, radius=4mm]-- cycle;
\coordinate [label=above:$A$] (B) at (0.08,0);
\coordinate [label=below:$B$] (B) at (1.5mm,-2.6mm);
\coordinate [label=below:$C$] (B) at (-0.5mm,-2.6mm);
\end{tikzpicture}
\footnotesize
не равен
\tiny
\begin{tikzpicture}[scale=3]

  \filldraw[fill=black,draw=black] (0.08,0) -- (1.1mm,-2.6mm)
      arc [start angle=-79, end angle=-105, radius=4mm]-- cycle;
\coordinate [label=above:$D$] (B) at (0.08,0);
\coordinate [label=below:$E$] (B) at (1.5mm,-2.6mm);
\coordinate [label=below:$F$] (B) at (-1mm,-2.6mm);
\end{tikzpicture}
\footnotesize
\\поскольку, если
\tiny
\begin{tikzpicture}[scale=3]

  \filldraw[fill=yellow,draw=yellow] (0.08,0) -- (1.5mm,-2.6mm)
      arc [start angle=-79, end angle=-105, radius=4mm]-- cycle;
\coordinate [label=above:$A$] (B) at (0.08,0);
\coordinate [label=below:$B$] (B) at (1.5mm,-2.6mm);
\coordinate [label=below:$C$] (B) at (-0.5mm,-2.6mm);
\end{tikzpicture}
\footnotesize
=
\tiny
\begin{tikzpicture}[scale=3]

  \filldraw[fill=black,draw=black] (0.08,0) -- (1.1mm,-2.6mm)
      arc [start angle=-79, end angle=-105, radius=4mm]-- cycle;
\coordinate [label=above:$D$] (B) at (0.08,0);
\coordinate [label=below:$E$] (B) at (1.5mm,-2.6mm);
\coordinate [label=below:$F$] (B) at (-1mm,-2.6mm);
\end{tikzpicture}
\footnotesize
, то
\\\begin{tikzpicture}[line width=2pt]

\coordinate [label=above:$C$] (A) at (0,0);
\coordinate [label=above:$B$] (B) at (1, 0);

\draw [-, black] (A) -- (B);

\end{tikzpicture}
=
\begin{tikzpicture}[line width=1.5pt]

\coordinate [label=above:$F$] (A) at (0,0);
\coordinate [label=above:$E$] (B) at (1, 0);

\draw [-, yellow] (A) -- (B);

\end{tikzpicture}
(пр. I.4),
\\что противоречит гипотезе;
\tiny
\\\begin{tikzpicture}[scale=3]

  \filldraw[fill=yellow,draw=yellow] (0.08,0) -- (1.5mm,-2.6mm)
      arc [start angle=-79, end angle=-105, radius=4mm]-- cycle;
\coordinate [label=above:$A$] (B) at (0.08,0);
\coordinate [label=below:$B$] (B) at (1.5mm,-2.6mm);
\coordinate [label=below:$C$] (B) at (-0.5mm,-2.6mm);
\end{tikzpicture}
\footnotesize
не меньше
\tiny
\begin{tikzpicture}[scale=3]

  \filldraw[fill=black,draw=black] (0.08,0) -- (1.1mm,-2.6mm)
      arc [start angle=-79, end angle=-105, radius=4mm]-- cycle;
\coordinate [label=above:$D$] (B) at (0.08,0);
\coordinate [label=below:$E$] (B) at (1.5mm,-2.6mm);
\coordinate [label=below:$F$] (B) at (-1mm,-2.6mm);
\end{tikzpicture}
\footnotesize
\\поскольку если
\tiny
\begin{tikzpicture}[scale=3]

  \filldraw[fill=yellow,draw=yellow] (0.08,0) -- (1.5mm,-2.6mm)
      arc [start angle=-79, end angle=-105, radius=4mm]-- cycle;
\coordinate [label=above:$A$] (B) at (0.08,0);
\coordinate [label=below:$B$] (B) at (1.5mm,-2.6mm);
\coordinate [label=below:$C$] (B) at (-0.5mm,-2.6mm);
\end{tikzpicture}
\footnotesize
<
\tiny
\begin{tikzpicture}[scale=3]

  \filldraw[fill=black,draw=black] (0.08,0) -- (1.1mm,-2.6mm)
      arc [start angle=-79, end angle=-105, radius=4mm]-- cycle;
\coordinate [label=above:$D$] (B) at (0.08,0);
\coordinate [label=below:$E$] (B) at (1.5mm,-2.6mm);
\coordinate [label=below:$F$] (B) at (-1mm,-2.6mm);
\end{tikzpicture}
\footnotesize
, \\то
\begin{tikzpicture}[line width=2pt]

\coordinate [label=above:$C$] (A) at (0,0);
\coordinate [label=above:$B$] (B) at (1, 0);

\draw [-, black] (A) -- (B);

\end{tikzpicture}
=
\begin{tikzpicture}[line width=1.5pt]

\coordinate [label=above:$F$] (A) at (0,0);
\coordinate [label=above:$E$] (B) at (1, 0);

\draw [-, yellow] (A) -- (B);

\end{tikzpicture}
(пр. I.4),

что противоречит гипотезе;

\tiny
\begin{tikzpicture}[scale=3]

  \filldraw[fill=yellow,draw=yellow] (0.08,0) -- (1.5mm,-2.6mm)
      arc [start angle=-79, end angle=-105, radius=4mm]-- cycle;
\coordinate [label=above:$A$] (B) at (0.08,0);
\coordinate [label=below:$B$] (B) at (1.5mm,-2.6mm);
\coordinate [label=below:$C$] (B) at (-0.5mm,-2.6mm);
\end{tikzpicture}
\footnotesize
>
\tiny
\begin{tikzpicture}[scale=3]

  \filldraw[fill=black,draw=black] (0.08,0) -- (1.1mm,-2.6mm)
      arc [start angle=-79, end angle=-105, radius=4mm]-- cycle;
\coordinate [label=above:$D$] (B) at (0.08,0);
\coordinate [label=below:$E$] (B) at (1.5mm,-2.6mm);
\coordinate [label=below:$F$] (B) at (-1mm,-2.6mm);
\end{tikzpicture}
\footnotesize
\flushright
ч.т.д


\begin{tikzpicture}[line width=2pt]


\draw (0,0) coordinate (A) -- (1.5,-5) coordinate (B)
         -- (-3,-6) coordinate (C) --cycle
           pic [fill=yellow, angle radius=1cm]                      {angle = C--A--B};
\draw [-,red] (B) -- (A) -- (C) -- (B);
\draw [-, cyan,line join =mitter] (-0.03, -0.06) -- (A) -- (B);
\draw [-, black, line join =mitter] (-2.97, -5.94)--(C) -- (B) -- (1.48, -4.95);
\coordinate [label=above:$A$] (A) at (0,0);
\coordinate [label=below:$B$] (B) at (1.5,-5);
\coordinate [label=below:$C$] (C) at (-3,-6);


\end{tikzpicture}
\\
\begin{tikzpicture}[line width=1.5pt]


\draw (0,0) coordinate (A) -- (1.5,-6) coordinate (B)
         -- (-3,-7) coordinate (C) --cycle
           pic [fill=black, angle radius=1cm]                      {angle = C--A--B};
\draw [-,red] (B) -- (A) -- (C) -- (B);
\draw [-, cyan,line join =mitter] (-0.003, -0.006) -- (A) -- (B);
\draw [-, yellow, line join =mitter] (-2.997, -6.993)--(C) -- (B) -- (1.498, -5.994);
\coordinate [label=above:$A$] (A) at (0,0);
\coordinate [label=below:$B$] (B) at (1.5,-6);
\coordinate [label=below:$C$] (C) at (-3,-7);
\end{tikzpicture}
\end{multicols}
\end{document}